\documentclass{article}
\usepackage{graphicx} 
\usepackage{url}
\usepackage{amsmath}


\title{\vspace{-4cm}Stochastic Simulation Group Project 2025}

\begin{document}

\maketitle

\section{Introduction}
While stochastic simulation as a whole is hard to define,
the goal in any simulation study is to estimate ($\hat{\theta}_i$), one or more parameters 
\begin{align}
    \theta_i = E\phi(X) \label{eq: expectation of loss function}
\end{align} 
where function $\phi(X)$ encodes the quanitity or relationship we aim to evaluate 
for some domain $X$. \cite{alma9946168020001381}.
\\
In a statistical application, $X$ is commonly a random variable and $\phi(X)$ will be
some statistical model we want to evaluate.
However, $X$ may also be deterministic, but $\phi$ is of a complex enough nature
that a simulated stochastic model is preferred over analytical or numerical approaches.
\\ \\
In this chapter we delve into how stochastic simulation is used to efficiently model
light particles in computational graphics, a process known as path tracing.

\section{Path Tracing Model}
Fig() shows how we model our system as a 3D Cartesian coordinate space, with a 
hemispherical global light source with intensity $L_e$. As light moves through the 
space with position $\vec{x}$, it interacts with diffuse surfaces 
represented by an area $A_n$ at bounce $n$.
Each recursive interaction results in a bounce at position $\vec{x}_n$, 
where the light is reflected toward $\vec{x}_{n+1}$, with a probability distribution function 
$f_{r}(\vec{x}_{n-1},\vec{x}_{n},\vec{x}_{n+1})$. The intensity is scaled by
$0$ if the path is impossible, and $1$ if it's possible, calculated by the
geometry factor $G(\vec{x}_{n-1},\vec{x}_n)$.\\
Thus after bounce $n$, the light will leave with intensity 
\begin{align}
    L_n(\vec{x}_n,\vec{x}_{n+1}) = 
    \int_{A} L_{n-1}(\vec{x}_{n-1},\vec{x}_n)G(\vec{x}_{n-1},\vec{x}_n)
    f_r(\vec{x}_{n-1},\vec{x}_n,\vec{x}_{n+1})\,dA_n
\end{align}
representing all the possible paths that could contribute to $L_n$.
\\
\\
Now suppose you want to calculate what a virtual pinhole camera would see inside this model.
If each pixel has direction from the pinhole $\vec{\omega}_i$, at position $\vec{x}_i$.
Then by summing the intensity $L_{n}$,
of all the possible paths of light that have direction $\vec{\omega}_i$ towards
position $\vec{x}_i$ after $n$ bounces.
\begin{align}
    L_i(\vec{x},\vec{\omega}_i) = \sum_{n=0}^{\infty}L_n(\vec{x},\vec{\omega}_i).
\end{align}
we get the brightness of pixel $i$. Do this for every pixel and you have an image.
\\
However, to solve this we have an infinite sum of a recursive integral with dimensionality $2n$.
At first glance this equation might seem impossible to solve. However, we will see in this chapter how
stochastic simulation allows us to easily evaluate this integral to a high
precision. \cite{KajiyaJamesT.1986Tre} 

\section{Generating Random Numbers}
In any simulation study, the first step in producing a random variable with our desired 
distribution, is producing a sequence of random numbers.
While there are many definitions of randomness that change with the application or 
philosophy.
In simulation theory we define randomness as events which are non-deterministic 
and also follow a known distribution \cite{shiryaev2016probability}. 
Although, knowing the distribution doesn't seem random at first glance. 
But much like how a fair coin follow a $X \sim \text{Bin}(\frac{1}{2})$ distribution
yet has a random chance of landing on heads or tails. Randomness does not assume we 
cannot make predictions about a large sample of events.

\subsection{True Random Numbers}
One subset of random numbers are true random numbers, these are what first come to 
mind at the mention of randomness, and are in every essence truly non-deterministic.
Ideally, all random number generators (RNG) would produce such stochastic numbers, 
however most of these process measure naturally entropic phenomena and are thus
slow and unsuited to most computer programs.\\
One such example would be to measure the time $s_1,s_2,\dots,$ between events of atomic decay,
and let 
\begin{align}
 X =
    \begin{cases} 
      1 & \text{if } x_{2i-1}<x_{2i} \\ 
      0 & \text{otherwise}
    \end{cases}
\end{align}
Then $X$ follows a 
\begin{align*}
    X \sim \text{Bern}(0.5).
\end{align*}
distribution.
Alternatively, we can use the HAVEGE ("HArdware Volatile Entropy Gathering and Expansion”)
\label{point: HAVEGE} system,
that produces random numbers based off your hardwares internal state \cite{gentle2003random}. 
\\
However, this has no known distribution and is also relatively slow and thus not useful for sampling.
Because of these many limitations on true random numbers we usually opt to
use deterministic algorithms known as a pseudo-random number generators (PNG).

\subsection{Pseudo Random Numbers}
Pseudo-random numbers are completely deterministic and with the correct 'seed' $x_0$
can be completely predicted. This makes them unsuitable for any application that requires 
high security such as cryptography. However, in simulation we only need the appears that they
are random and independent and this can be achieved with a suitably good generator.

\subsubsection{Linear Congruential Generators}
By creating a true random seed $x_0$ through HAVEGE from \ref{point: HAVEGE},
we can use the algorithm 
\begin{align} 
    \label{eq: Linear Congruential Generator}
    x_{i} \equiv (ax_{i-1} + c) \pmod{m},\quad\text{    with    }\quad0\leq x_{i}<m
\end{align}
to recursively produce a 'Lehmer' sequence of uniformly distributed numbers. 
We call $a$ the multiplier, $c$ the increment and $m$ the modulus of the generator.
Often, $c$ is taken to be $0$ for simplicity, giving us a multiplicative congruential 
generator.
Although the sequence is clearly deterministic, if we scale it onto the interval 
$(0,1)$ by dividing through by $m$. It can be seen in fig() to be a random variable
$X\sim(0,1)$. \cite{alma9954732790001381}
\\
\\
For every generator, there will exist applications in which the determinism of the generator
is exploited to introduce bias.
Hence, for a given generator we must choose parameters that maintain the perceived independence
and identical distribution of our random variable. 

\subsubsection{Period}
It can be seen from Eq(\ref{eq: Linear Congruential Generator}), that the
Lehemer sequence will repeat with period no greater than $m$. An important parameter
to account for is the periocidy of your PNG. It is generally accepted, that for linear methods
an upper bound for the sample size is $n = \sqrt{k}$, where $k$ is the period of the PNG \cite{HELLEKALEK1998485}.
\\
It turn out that the maximum period can be solved using the Euler totient function $\phi(m)$.
Additionally, it can be seen from Eq(\ref{eq: Linear Congruential Generator}),
the period  of a multiplicative generator with multiplier $a$, is the smallest value
of $k$ such that 
\begin{align}
    &x_{i+k}=x_{i} \notag
    \intertext{where} 
    &x_{i+k}\equiv x_{i} \pmod{m}\notag
    \intertext{solving recursively we have} 
    &x_{i+k}=a^{k}x_{i} \pmod{m}.\notag
    \intertext{Thus} &a^{k}\equiv 1\text{ mod }m
\end{align}
where $a$ is the 'primitive root modulo $m$', when $k=\phi(m)$.
In most applications, a period of at least the order of $10^{9}$ is required. \cite{gentle2003random}

\subsubsection{Independence through Dimensions}
Additionally, we must ensure independence is maintained through the dimensions of our applicaiton.
In our example problem of path tracing, it can be seen that for any given path with $n$ bounces,
we have dimensionality of $2n$. However, it can be seen in fig() that for a generator with
parameters that maximise the period, the distribution of $d$-tuples $x_n=(x_{n},x_{n+1}\dots x_{n+d})$
with dimensions $d$ does not necessarily have a good lattice structure. This suggests a correlation in 
higher dimensions, and will thus introduce bias in applications such as our example.
To prevent this emperical testing such as the spectral test should be conducted to 
ensure that there is independence within the required dimensions. \cite{HELLEKALEK1998485}

\section{Generating Random Variables}
Often a chosen generator will not match our desired distribution and thus we need
methods to simulate random variables with different distributions. Many such methods
exist but one particularly suited for when using a linear generator
is the inverse transformation method.  

\subsection{The Inverse Transformation Method}
Let U be a uniform (0,1) random variable. If we define the random variable X by
\begin{align}
	X=F^{-1}(U)
\end{align}
where $F$ is a c.d.f  and 
\begin{align}
F^{-1}(u)=\text{inf}\{x:F(x)\geq u\},\quad\quad 0\leq u\leq 1.
\end{align}
Then $X$ has c.d.f $F$.
\\
This means for our random variable from any uniform generator we can then simulate a distribution that has a known probability function. 
\\
This works since $F(x)$ is a c.d.f and hence monotone, and so 
\begin{align}
F^{-1}(U)\leq a &\iff U\leq F(a).\notag
\intertext{Thus we have}
F_{X}(a)&=P\{X\leq a\} \\ \notag
&=P\{F^{-1}(U)\leq a\} \\ \notag
&=P\{U\leq F(a)\}\\ \notag
&=F(a) \quad \cite{ross2007introduction}. 
\end{align}
It should be noted that the theory for these equations assumes the randomness of 
any distribution function. Hence, when using PNGs we should take care to ensure that
the algorithm isn't exploiting biases in our generator \cite{alma9946168020001381}.

\subsection{Good-fit Tests}

While a good-fit emperical test cannot test for independence, if our sample can pass
this test for a given distribution $f_X$. It can thereafter be considered a random variable 
$X$ with distribution funciton $f_X$. \cite{HELLEKALEK1998485}.  
\\
The Glivenko-Cantelli theorem which states that,
for a random sample  $X_{1},\dots,X_{n}$ from some unknown c.d.f. $F_X(x)$,
and our 'sample c.d.f.' $F_N(x)$, defined as 
\begin{align}
F_{N}(x)=\frac{1}{N}(\text{number of $X_{i}$ less than or equal to $x$}) ~,
\end{align}
then
\begin{align}
\lim_{ N \to \infty } P\left[\sup_{x} |F_{N}(x)-F_{X}(x)|>\epsilon\right]  = 0 ~.
\end{align}
Thus showing for large N the maximum deviation
$\sup_{x} |F_{N}(x)-F_{X}(x)|$,
denoted simply as $D_N$, between the 'true' function $F_X$ and our sample c.d.f. $F_N$ is small 
for all $x$. 
Using this quality Kolmogrov proved and Smirnov later formalised that for any continuous
distribution $F_X$
\begin{align}
\lim_{ N \to \infty } P(\sqrt{ N }~D_{N}\leq x)=H(x)
\end{align}
where $H(x)$ is a tabulated function. 
\\
Thus remarkably this Kolmogrov-Smirnov test allows a sample of any distribution and any 
sufficiently large sample size (at least 35) to be evaluated against $H(x)$. Such 
a test is known as non-parametric and has a broad application. \cite{alma9954732790001381}

\section{Monte Carlo Integration}
From Eq.(\ref{eq: expectation of loss function}) it is not immediately evident 
how a stochastic simulation can be used to evaluate an integral $\theta$ on the domain $D$
\begin{align}
    \theta &= \int_{D} g(x) \, dx. \notag \\
    \intertext{However, if we transform it to be an expected value for some random variable} \notag 
    &= \int_{D} \frac{g(x)}{f_X(x)}f_X(x) \, dx \quad\quad \text{    s.t.    } f_X(x) >0 \text{    when    } g(x)\neq0, \notag\\ 
    \intertext{where $f_X$ is the p.d.f. of $X$ and so,} \notag \\
    &=E \left[ \frac{g(X)}{f_X(X)} \right].
\end{align}
It can be seen that this is in the form of Eq.(\ref{eq: expectation of loss function}),
where $\theta$ is the parameter we are trying to solve for. 
Hence, we have now setup this deterministic integral as a stochastic model.
Thus, in the same way we would for a stochastic model, we can solve this through random sampling. \\
In the simplest case where $D = [a,b]$ and $X\sim U(a,b)$ we have 
\begin{align}
    \theta = (b-a)E[g(X)] 
\end{align}
Taking the sample mean as an unbiased estimator
\begin{align}
    \hat{\theta} = (b-a) \frac{1}{N} \sum_{i=1}^{N}g(X_i)
\end{align}
$\theta$ can be solved for $N \rightarrow \infty$.

\section{Importance Sampling}
Importance sampling is a method of variance reduction that reduces the burden for 











\bibliographystyle{plain}
\bibliography{references} 
\end{document}